% Options for packages loaded elsewhere
\PassOptionsToPackage{unicode}{hyperref}
\PassOptionsToPackage{hyphens}{url}
%
\documentclass[
]{article}
\usepackage{lmodern}
\usepackage{amssymb,amsmath}
\usepackage{ifxetex,ifluatex}
\ifnum 0\ifxetex 1\fi\ifluatex 1\fi=0 % if pdftex
  \usepackage[T1]{fontenc}
  \usepackage[utf8]{inputenc}
  \usepackage{textcomp} % provide euro and other symbols
\else % if luatex or xetex
  \usepackage{unicode-math}
  \defaultfontfeatures{Scale=MatchLowercase}
  \defaultfontfeatures[\rmfamily]{Ligatures=TeX,Scale=1}
\fi
% Use upquote if available, for straight quotes in verbatim environments
\IfFileExists{upquote.sty}{\usepackage{upquote}}{}
\IfFileExists{microtype.sty}{% use microtype if available
  \usepackage[]{microtype}
  \UseMicrotypeSet[protrusion]{basicmath} % disable protrusion for tt fonts
}{}
\makeatletter
\@ifundefined{KOMAClassName}{% if non-KOMA class
  \IfFileExists{parskip.sty}{%
    \usepackage{parskip}
  }{% else
    \setlength{\parindent}{0pt}
    \setlength{\parskip}{6pt plus 2pt minus 1pt}}
}{% if KOMA class
  \KOMAoptions{parskip=half}}
\makeatother
\usepackage{xcolor}
\IfFileExists{xurl.sty}{\usepackage{xurl}}{} % add URL line breaks if available
\IfFileExists{bookmark.sty}{\usepackage{bookmark}}{\usepackage{hyperref}}
\hypersetup{
  hidelinks,
  pdfcreator={LaTeX via pandoc}}
\urlstyle{same} % disable monospaced font for URLs
\usepackage{graphicx}
\makeatletter
\def\maxwidth{\ifdim\Gin@nat@width>\linewidth\linewidth\else\Gin@nat@width\fi}
\def\maxheight{\ifdim\Gin@nat@height>\textheight\textheight\else\Gin@nat@height\fi}
\makeatother
% Scale images if necessary, so that they will not overflow the page
% margins by default, and it is still possible to overwrite the defaults
% using explicit options in \includegraphics[width, height, ...]{}
\setkeys{Gin}{width=\maxwidth,height=\maxheight,keepaspectratio}
% Set default figure placement to htbp
\makeatletter
\def\fps@figure{htbp}
\makeatother
\setlength{\emergencystretch}{3em} % prevent overfull lines
\providecommand{\tightlist}{%
  \setlength{\itemsep}{0pt}\setlength{\parskip}{0pt}}
\setcounter{secnumdepth}{-\maxdimen} % remove section numbering

\author{}
\date{}

\begin{document}

\textbf{USE OF IoT IN AGRICULTURE}

Submitted by:

Tejal Singh

AI\&DS-B

21011101133

Introduction:

IoT (Internet of Things) architecture is being increasingly used in
agriculture to improve crop yields and reduce costs. IoT devices such as
sensors, cameras, and drones are used to collect data on soil moisture,
temperature, and other environmental factors. This data is then analyzed
by machine learning algorithms to optimize irrigation, fertilization,
and other key aspects of crop management.

One key area where IoT is being used in agriculture is precision
farming. Precision farming involves using precision agriculture tools
such as GPS, sensors, and drones to collect data on crop health, soil
conditions, and other factors that affect crop growth. This data is then
analyzed to optimize planting, irrigation, fertilization, and other key
aspects of crop management.

IoT also plays a key role in Livestock monitoring, IoT enabled devices
such as temperature sensors, cameras, and weight sensors can be placed
on cows and other livestock. This data can be used to monitor the health
of the animals, and ensure they are being raised in a safe and healthy
environment.

Another area where IoT is being used in agriculture is in the use of
autonomous vehicles. IoT-enabled tractors, drones, and other vehicles
are used to plant, harvest, and transport crops. This allows farmers to
reduce labor costs and improve crop yields.

Architecture:

There are usually four layers present in an IoT system, which are as
follows:

\textbf{{1.Perception Layer}} : Any IoT system's initial layer is made
up of "things" or

endpoint devices that act as a link between the real world and the
digital one. The physical layer, which contains sensors and actuators
capable of gathering, receiving, and processing data across a network,
is referred to as perception. Wireless or wired connections can be used
to link sensors and actuators. The components' range and locations are
not constrained by the design.

In the agriculture sector IoT sensor nodes collect information from the
farming environment, such as soil moisture, air humidity, temperature,
nutrient ingredients of soil, pest images, and water quality, then
transmit collected data to IoT backhaul devices. Depending on the
operation purpose and installation location, IoT sensor nodes can be
installed as RFDs (reduced-function devices), which only communicate
with FFDs (full-function devices). These nodes cannot communicate with
the other RFDs, aiming to save energy and decrease investment costs.

\textbf{{2.Network Layer}}: An overview of the data flow across the
programme is given by the network layers. Data Acquiring Systems (DAS)
and Internet/Network gateways are present in this tier. Data aggregation
and conversion tasks are carried out by a DAS (collecting and
aggregating data from sensors, then converting analogue data to digital
data, etc.). Data gathered by the sensor devices must be sent and
processed. The network layer performs that function. It enables
connections and communication between these gadgets and other servers,
smart gadgets, and network gadgets. Additionally, it manages each
device's data transfer.

The network layer in the agriculture sector works as following,the
sensor devices are connected together via either long-range or
short-range wireless networks, such as 4G, Wi-Fi, Bluetooth LE, ZigBee,
and others. The system can be implemented using embedded systems, which
are made up of tiny modules that provide a network and aid in delivering
data more securely and accurately. The agricultural system would
function more effectively in this way. The information about several
distinct parameters, such as soil moisture, temperature, humidity, air
quality, and ground water level regarding a given location, will be
transmitted via a tiny embedded device in this smart agricultural
system. As a result, the network layer is essential for data
transmission from sensor devices to software or databases, where
computations are carried out using the sent data.

\includegraphics[width=5.30417in,height=5.6in]{media/image1.gif}

3.\textbf{{Processing / Middleware Layer :}} The IoT ecosystem's
processing layer functions as its brain. Before being transported to the
data centre, data is often evaluated, pre-processed, and stored here. It
is then retrieved by software programmes that handle the data and plan
future actions. This is where edge analytics or edge IT comes into play.

In this sector, a remote database is used to store the substantial
quantity of data that is received from the remote terminal unit in the
perception layer. On a timely basis, a predictive analytic algorithm is
applied to the data. These projected values are used to execute the
required block of instructions. The instructions mention things like
automatically watering the crops and automatically giving them shade.
These instructions are in charge of altering the system's application
gateway. Given that the weather in the globe is continuously changing
and that the farm is hoping for the finest output possible, analytical
operations play a significant role. The user may find it helpful to
control crop production using these actions based on constantly changing
real-time data.

4. \textbf{{Application Layer:}} The final layer in the system is the
application layer, it comprises of~communication protocols and hosting
interface techniques for a communication network, including MQTT, AMQP,
CoAP, etc. A website or an Android application, for example, receives
data from the middleware layer that has been encrypted; these devices
then employ internal software to decode the data and store it locally
for later use. This information is used to demonstrate the current
trend. The programme keeps track of data and makes wise recommendations
to progressively boost the system's throughput.

Conclusion:

The idea of fusing agriculture with cutting-edge technology may be
realised by incorporating technology into the traditional and archaic
methods at the core of agriculture. The IoT-based strategy makes it
simpler for technology to assess environmental factors like climate,
soil quality, etc. while the software layer strengthens the conventional
wisdom of the elderly. Technology must take lessons from existing
methods and automate jobs to make life easier for people. Due to its
flexibility to adapt to changing field circumstances, this technology
has the potential to become worldwide. Therefore, IoT might assist
achieve smart farming and save a drastically declining agricultural
business by streamlining farming practises.

\end{document}
